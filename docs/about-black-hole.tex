\section{统一技术设定论述,关于超维度出入口(即虫洞)}

\label{sec:DOC-about.black.hole}

\tred{+ 查看不必要的科学文献}

\tred{– hide block}

\GG{\tred{统一技术设定论述,关于超维度出入口(即虫洞)的分级}}

\hr

\bb{目的:}量化并整体上把超维度出入口理论联系起来,基于实际使用和作品情况指派相应的名称,为基金会写作带来方便,也为作者们提供一个量子叠加态方面的设定。\footnote{也就是说,它既\ii{是}设定,也\ii{不}是设定。}

\bb{简介:}本质上,我想提出一个参考系统,能把当前出现过的超维度出入口松散地联系起来,或者帮助更好地理解它们。本篇论述将解释名称,并适时举例说明。我会尝试对每一个写入过数据库的虫洞进行分类,并在此套系统下根据它们传输物质的效率,收容方面起的作用(参照斯克兰顿现实稳定锚),以及其他附属任务进行分级。

\ii{作者注:本文档使用了几个已建立的基金会宇宙伪设定,如果写手赞同的话,可以直接用到作品中。这些名称大多都是纯理论假设的,所以可以更改。}

\hr

\Gg{\bb{A级:}“相对论完整性”虫洞}

A级虫洞很适合基金会传送工作使用。它在折叠时空方面涉及的理论领域完全可以使用标准相对论物理模型描述。

这种虫洞包括自然出现的虫洞,其内部量子场可以被稳定在一定数值以允许以太自由流动。同样地,那些对基金会行动机动性有极大帮助的异常区域也可以归到其中。

A级虫洞示例:

\begin{itemize}
\item \hyperref[chap:SCP-1322]{SCP-1322}\footnote{如果它能被基金会完全控制的话;据我所知,目前没有Thaumiel等级虫洞存在(适时而变)。}
\item \hyperref[chap:SCP-120]{SCP-120}
\end{itemize}

\Gg{\bb{B级:}“信息高速路”虫洞"}

B级虫洞可以让物质以存储的信息方式运输\footnote{一般来说,这意味着B级虫洞是自然形成的。该过程通过“超填”量子储存塔到大于10\textsuperscript{69}bits\slash m\textsuperscript{2}的密度完成,其将使得媒介物坍缩为一个黑洞,并可通过后期修改来实现数据到数字或模拟现实的转换。更多信息请参见\hyperref[chap:]{信息是基础的吗?}。}。该过程可用于创建有形的网络接口,以实现数据的物理交互,与已构建地点之间的传输。

B级虫洞示例:

\begin{itemize}
\item \hyperref[chap:SCP-1549]{SCP-1549}中有提及
\end{itemize}

\Gg{\bb{C级:}“破碎入口”虫洞"}

此类虫洞为一不稳定的时空折叠,无法将物质传送到指定位置。在其出现\footnote{一种人为时空操纵的可能后果,因为理论认为虚无维度是两个不相关维度上的点在发生超维重叠时时空为抵消其不可能性而产生的自平衡现象。}时会产生小的虚无维度空泡,如控制不当可能会产生危害。这些虫洞无法预测,会喷射出未知来源的物质。

C级虫洞示例:

\begin{itemize}
\item \hyperref[chap:SCP-3001]{SCP-3001}中有提及
\end{itemize}

\Gg{\bb{D级:}“非回溯”虫洞"}

D级虫洞是基础的销毁单元,其通向非宇宙。它们是时空曲率的标量恒定处,指空间中不存在现实与物理定律的点。\footnote{这到底会形成一个黑洞还是使得此虫洞消失还有待观察。}

D级虫洞示例:

\begin{itemize}
\item \hyperref[chap:SCP-123]{SCP-123}
\end{itemize}

\Gg{\bb{E级:}“因果暂失”虫洞"}

E级虫洞定义了许多不可能存在的超维区域,因此对现实存在有异常影响,或者无法以对基金会有意义的方式创建利用。它们要么已经消失,只是昙花一现且没有明显的(也可能是异常的)原因,要么以违反自然法则的方式存在。通常,这些出入口是在固定位置由异常手段创建的,并且对基金会的行动没有任何益处,甚至需要进行分级和收容。

E级虫洞示例:

\begin{itemize}
\item \hyperref[chap:SCP-3321]{SCP-3321}
\item \hyperref[chap:SCP-3221]{SCP-3221}
\item \hyperref[chap:SCP-1437]{SCP-1437}
\item \hyperref[chap:SCP-2510]{SCP-2510}
\end{itemize}

还有更多,我敢肯定。

\GG{\tred{为什么?}}

这是个大问题,是吧?为什么这个?为什么那个?为什么笔耕不辍?为什么搞出了休谟?

我喜欢我虚无的格式塔组织有统一的术语。那能让我觉得心中很快乐,也巩固了基金会的临床腔结构。

但大家还是要记得:

\ii{“没有什么设定。”}
