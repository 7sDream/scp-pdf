\chapter[SCP-037 白矮星]{
    SCP-037 Dwarf Star\\
    SCP-037 白矮星
}

\label{chap:SCP-037}

\bb{项目编号:} SCP-037

\bb{项目等级:} Euclid

\bb{特殊收容措施:} SCP-037以磁场收容于Site-32。 它被置于一个以导热,抗辐射的聚合物构成,且完全气密的小型地下室中。对象释出的热量辐射至周围的岩层中。当墙体变得脆弱时,在修复之前应急系统将生成一个氩等离子力场。当意外事故发生时{[}拒绝访问]。

\bb{描述:} 该物体发现于19██年的波佛特海的地磁北极附近。加拿大军事站点(CFS)在警报中报告有强烈的电磁干扰,随后一个极其明亮的物体从天空降向大海。驱逐舰SCPS监护者号予以响应,并且发现对象在水面上以一个不稳定的轨道不定地摇摆。施加收容措施后,对象立即被送至Site-32进行研究。

SCP-037似乎是一颗直径约5厘米(2英寸)的恒星。具有太阳表面1*10\textsuperscript{-12}倍的亮度,表面温度为5000K(通过UBVRI分析测定)。SCP-037的年龄及来源不明,但它的核反应被小心地监控起来以防不稳定情况。虽然并不知道它是否适用于已知的恒星形成和老化理论,光谱分析和对比得知它仍然是一颗典型的(除了其并不典型的尺寸以外)正在转变为红巨星的天体。它被认为是通过地球的磁北极进入地磁气圈的。

SCP-037之收容及运输方式可以通过强力的电磁铁来完成,设备的工件会通过自身的磁场自行排列整合。 目前为止收容的主要挑战一直是其极强的电磁辐射,其强到可以在地球高空轨道上以肉眼观测。它当前的外壳位于地下深处以防止被目击和帮助辐射逸散到周围的基岩里。事实上,整个设施及其周围的岩层就是一个巨大的散热器。

经过██年的研究,该恒星已处于电磁辐射变动中,表明其恒星演化过程大大加快了。如果参照标准恒星模型,这将很快引发其半径膨大100-300倍,以及随之而来的辐射能量变化。 对于该可能性,正在研究紧急突发事件的控制。恒星生命周期进一步发展的结果很可能是一次新星爆炸, 估计其能量可能有███████████。推断预测它可能在███████████发生。在此之前将阻止该过程或将其运输到地球的安全距离外的研究已经在起步中。
