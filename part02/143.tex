\chapter[SCP-143 刃木之森]{
    SCP-143 The Bladewood Grove\\
    SCP-143 刃木之森
}

\label{chap:SCP-143}

\bb{项目编号:}SCP-143

\bb{项目等级:}Euclid

\bb{特殊收容措施:}SCP-143被收容于毗邻第12号生态研究地域(Bio-Research Area-12)的一块超过2平方公里的地域。周边20公里内可目视SCP-143的山峰均禁止公众入内。除非当地降雨量足够,否则SCP-143每天通过一个大型喷淋系统定期浇灌2次。没有4级人员的许可禁止人员进入围场内,并建议不要碰触SCP-143或在未装备恰当防护装备时站在它们下面。非常重要的一点是,在SCP-143开始脱落时收容区域应保持无人进入。

\bb{描述:}SCP-143是一圃共300棵的树种独一无二的树木。数目外表上类似染井吉野樱(日本樱花)它们无法结出果实,而唯一已知繁殖方式是通过“根扦插”来从老样本上繁殖新树苗。

其花瓣呈现为淡淡的粉红色,轻微半透明,并且具有光滑玻璃的质感。必须小心处理花瓣,由于其边缘如日本刀般锋利,如果不慎很容易切开肌肤。

其树干呈现为浅灰色,并且有符合预想的木纹,虽然木纹摸上去非常光滑。然而,这些树木的木材和花瓣比绝大多数人造或天然物质坚硬,达到5000HB布氏硬度、承受温度高达1800°C,重量强度比甚至超过钛,比铝轻15\%。尽管达到这种硬度,这些木材和花瓣仍像大部分树木和花瓣那样十分易弯曲。

这两者都是众所周知地难以加工,然而在高温下,达到1500°C时,原本分离的碎片能够融合在一起。它们能成为优秀的盔甲,盾牌和武器。由于此植物生长迟缓,虽然花瓣很有规律地每年从植株上脱落两次,材料收获仍比较缓慢。

\bb{附录143-1:}当下被种植在现场的树苗获得于日本大和省1905年一棵母树上。其母树由一个日本刀匠家族持有,他们说他们是一名传奇刀匠“天国”的后裔。他们将母树称为\ii{“刃木之树”}( 刃木の木立ち),或者“剑之森”。这就是基金会获取加工这些木材和花瓣成为有用物品的技术的由来。

原树仍位于日本,由日本政府拥有,并仍由那个家族照料。然而,政府已经否认了所有树木的存在,并且将其所有产品都保存于国内。

\bb{附录143-A:}\ii{我们今天因为SCP-143失去了3个工作人员。他们去收集掉在树林里的花瓣的前一天,突然一阵风刮起,从树上吹下大量的花瓣在他们四周。风还持续刮了一整天。我最好派遣一个清理人员,但仍在刮风且花瓣仍然在落下。等到风停下一两天我们必须收集剩余的部分。}
