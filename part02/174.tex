\chapter[SCP-174 腹语玩偶]{
    SCP-174 Ventriloquist's Dummy\\
    SCP-174 腹语玩偶
}

\label{chap:SCP-174}

\begin{figure}[H]
    \centering
    \includegraphics[width=0.5\linewidth]{images/SCP-174.jpg}
    \caption*{SCP-174}
\end{figure}

\bb{项目编号:}SCP-174

\bb{项目等级:}\dd{Safe} Euclid,详见174-A事件

\bb{特殊收容措施:}SCP-174被收容在位于Site-19的储存单元-07(Storage Unit-07)中。与其互动的研究实验要优先选择具备高心灵抵抗指数的人员,且必须由两(2)名熟悉该物件的4级人员进行批准,才可将之取出。所有与SCP-174项目有关的人员都要接受定期心理评估;对该项目表现出迷恋或是保护倾向的人,需被实施B级记忆消除,并监视72小时。

\ii{收容措施附录,20██年██月██日:}在事件174-A发生之后,SCP-174和储存单元-07的主隔间都设有全天候视频监控,出现任何异常活动必须立即报给A█████████博士。此外,SCP-174身上被装入了GPS定位追踪设备,即使该物件脱离了基金会的控制,也能被快速回收。

\bb{描述:}SCP-174是一件木制腹语玩偶,其头顶到脚尖的距离约莫21cm,身上有几处轻微损坏,穿着破烂的衣装。从该物件的设计风格及其保养情况判断,它被制造的时间大约处于20世纪早期。可以通过SCP-174内部的某种机械装置来操控它的眼睛和嘴巴。

报告显示,当实验人员用眼角余光注视SCP-174时,他们会发现玩偶用一种渴望或悲伤的表情直直地看着他们;当实验人员用目光直视SCP-174时,这种异常表现则不会发生;通过某种间接方式观察SCP-174,譬如一面镜子或是实时监控影像,似乎会增加上述异常现象发生的可能性。据那些在项目附近活动的人报告,他们普遍对SCP-174抱有一种怜悯或同情的心理,但无法解释这种感觉发自何处。长期接触会令这种赋予对象人性化的心理暗示加深,在某些情况下,那些心灵抵抗指数特别低的人员甚至会认为SCP-174就是个活生生的人(例如把玩偶当成婴儿一般抱在怀里)。若将这种异常状况告知当事人,则所有当事人都可恢复正常行为,但仅仅清醒了若干分钟,他们就会再一次陷入迷障。

那些将SCP-174持在手中的实验人员报告称,有一种想要与玩偶“交谈”的冲动。当被问及何故,他们常以对象“太孤单”需要陪伴作为回答。实验人员还会变得替SCP-174说话,同时操作玩偶的表情来配合表达。在替SCP-174说话时,实验人员会提高音调,模仿孩童声线。经过高敏麦克风录音测定,目前已确认玩偶本身并没有在说话,或者发出任何噪音。这种演绎行为几近完美,但与当事人是否具有表演经验并无关系。“对话”内容很快会变成玩偶与实验人员讨论它的情绪状态,内容往往与玩偶过去的遭遇有关;大部分情况下,这种讨论会演变成SCP-174向实验人员倾诉一段故事,一段围绕该玩偶如何被抛弃被虐待的辛酸血泪史。但没有哪段故事是重复的,也没有证据能够证明那些故事的真实性,任何一段也没有。研究人员推测,SCP-174可能具有低级心灵感应能力,每一段故事似乎都围绕着某个主旨,能引起当事人员的共鸣。

经过这一点,实验人员会变得极受SCP-174影响,并尝试“保护”它以至于生人勿近,他们会想尽一切办法阻止其他人靠得太近或是与玩偶互动;某些情况下,受影响者甚至不惜使用致命武力。实验人员经常将SCP-174喊作他们的“宝贝”,或其他以示亲密的称谓。哪怕是将当事人与玩偶分开,这种影响也会持续若干个小时;甚至出现过一类极端的情况,玩偶已与当事人分开了两个星期,其影响力依然存在。该类影响是否会减弱目前尚无定论,因为后继实验缺乏合情的理由,受影响超过两个星期的人员都已被例行处决。与SCP-174完全隔离的受影响人员会因为担心玩偶的安全而变得多疑且偏执,常导致精神崩溃,类似遭受了丧子之痛的母亲那般。B级或更强烈的记忆消除已被证明能够有效治疗这种强迫症以及大部分由此产生的心理创伤;然而,几乎所有经历了该治疗手段的实验人员都会抱怨他们精神上出现了情感缺失,并可能演化为抑郁症。

\bb{附录174-1:实验日志(视频录音记录)}\\
实验人员:D-14285;女性,21岁,没有暴力犯罪史。\\
研究主管:A█████████博士\\
地点:Site-19,收容隔间-A4(Containment cell-A4)(研究相关人员在双向镜后面进行观察)

D-14285被命令将SCP-174持于手上,该实验人员犹豫了一下之后如是照做。几秒钟后,实验人员与SCP-174开始进行平凡的对话。大约2分钟后,当事人问SCP-174“到底发生了什么?”,此时玩偶开始叙述一段故事,关于它是如何留在一座房子里,并于随后发生的火灾中受伤,然后被其故主无情抛弃。\ii{<注意:该实验人员的资料显示她正是一宗发生于19██年纵火案的受害者。>}当事人开始劝慰玩偶,用积极的语言安抚它。当SCP-174谈及它是多么地孤单,想要寻找伙伴,实验对象便开始用空余的拳头砸向大门。随后持枪警卫进入隔间内,实验对象立即跳回角落里,同时将玩偶圈在怀中并窃窃私语(具体内容未能被麦克风采集)。警卫成功将实验人员与SCP-174分开,然后离开隔间。此时实验人员大喊“他们夺走了它,我亲爱的宝贝!”,然后开始尝试逃跑,徒劳地对着大门拳打脚踢。

\ii{注:}此时A█████████博士命令实验结束。已尝试劝说D-14285平静下来,但收效甚微,该D级人员随后被处决(本实验尚属于对SCP-174测试的初始阶段,在当时记忆消除的功效并没有体现出来)。

\bb{附录174-2:174-A事件}\\
20██年██月██日,当A█████████博士进入储存单元-07,发现SCP-174就这么坐在它那隔间旁边的地板上,直勾勾地盯着主大门;这扇通往SCP-174隔间的大门早已被封闭,之前的几个星期里也没有任何开启记录。于是,在将该玩偶重新收容后,储存单元-07的内部便被装上了实时视频监控,同时SCP-174也被安置了GPS定位系统,一切都是为了防患于未然。
