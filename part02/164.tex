\chapter[SCP-164 乌贼肉瘤]{
    SCP-164 Squid Tumors\\
    SCP-164 乌贼肉瘤
}

\label{chap:SCP-164}

\bb{项目编号:}SCP-164

\bb{项目等级:}Euclid

\bb{特殊收容措施:}有关SCP-164的一切都需要用对抗3级生物危害的措施来收容,其冷冻生物容器单元需有明确标识,并维持在10摄氏度。作为一种病原体,SCP-164的传染性不是很高,当研究人员研究该病原体或是其感染物时,佩戴乳胶手套和口罩便可以完成有效隔离。任何不经意间被感染的人员在其首次症状出现后,都应进行6个月的化疗疗程,必要时还需动用外科手术。

一旦在人群中爆发感染,需采用Alef-█作为传播媒介来执行掩盖程序。

\bb{描述:}SCP-164是一种能导致宿主患上癌肉瘤的细胞。尽管这种细胞的DNA与人类相似,但它更像是一种无性繁殖的单细胞寄生虫。SCP-164有着显著的几个特点:

\begin{itemize}
\item SCP-164只会寄生(感染)人类,其传染途径有:血源、性接触、皮肤接触以及空气传播。化学疗法与外科手术是极其有效的治疗方式,适用于该疾病的任何阶段。
\item 由SCP-164产生的肿瘤体积超过某种临界点后,在75\%的情况下,其发展历程与普通癌肉瘤没什么区别。然而,有25\%的可能性,肿瘤会在宿主体内产生一个全新且独立 的 有机体。在患有多个肿瘤的情况下,其中一部分甚至全部都有可能产生这种变异。上述有机体会分化成合子(受精卵)并像胚胎那样开始复制。由于外表上难见端倪,这种变异在肿瘤初期很容易被忽视。
\end{itemize}

奇怪的是,这些有机体在成熟之后,会变成完全不同于原发癌肉瘤样的东西,更像是管鱿目乌贼科的某种未知生物。将这些有机体放入水中,它们适应良好,活动自如,可以实现猎食、防卫、生殖等行为。然而,存留宿主体内的上述有机体,依然保有肿瘤特性,它们几乎不会移动或是改变位置,而是继续按照一定规律生长,直至宿主死亡。该有机体(SCP-164-2)通常不为人们所知,往往在活检或是外科手术中才探查到有它们存在。

SCP-164有机体和肿瘤可能会与宿主产生有趣的生理反应,尤以下列情况为甚:

\begin{itemize}
\item 女性D级人员,23岁:SCP-164肿瘤细胞在子宫壁着床,宿主显然把它识别成了自体胚胎,孕育9个月后,正常产下活体SCP-164-2样本。
\item 男性D级人员,30岁:肿瘤细胞寄生了脊髓,对宿主的中枢神经系统产生了干扰与破坏。结果,SCP-164-2自身的动作偶尔会引起宿主双下肢产生矛盾运动(limbs to flail),似乎在两个生物体之间建立了某种“交叉线路”。活检支持这一控制假说。
\item 男性D级人员,25岁:肿瘤细胞坐落于食道和气管,其体积很快便能塞满这两条重要的生命通道,令宿主一命呜呼。然而,这些生长中的肿瘤却神奇地挪转到了颈部后方,在变异体积临界点到达前,避免宿主窒息死亡。█████博士认为,SCP-164可能经过了一番深思熟虑,才采取了这一行动。
\end{itemize}
