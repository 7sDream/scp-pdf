\chapter[SCP-183 “织音者”]{
    SCP-183 "Weaver"\\
    SCP-183 “织音者”
}

\label{chap:SCP-183}

\bb{项目编号:}SCP-183

\bb{项目等级:}Safe

\bb{特殊收容措施:}SCP-183的收容室之中必须隔音并且其中排列着重的钢板。收容室的尺寸并不重要,因为SCP-183没有表现出什么偏好并且会利用上每一寸的空间。强烈建议人员不
进入其收容室;因为这种行为是无意义并且极度危险的,并且任何尝试这种行为的人都会被认为是受到了未被观察到的SCP-183的歌声的心灵影响(见下)。收容室上的开口要小而简单,
最好是只为了喂食的目的而开的。

该生物没有表现出逃跑的真正欲望,但是这种情况如果真的发生了,它能马上使得整个研究设施寸步难行,并且会变得有着潜在的高危可能。SCP-183的歌声被描述为愉悦的,并且长久
以来没有观察到这歌声有着任何对于潜意识的影响,所以只要研究者们愿意,他们可以将在植入设施之中的麦克风保持打开的状态。

\bb{描述:}SCP-183是一个未知的生物学实体,并且以所有的手段进行探测都是隐形的。它存在的证据主要来自于很明显是它的组织之中织出来的单丝线。这些纤维极细并且非常耐用;
它们能轻而易举地以很小的力量从软组织、骨头甚至是防弹衣之中切割而过。并且因为伴有着难以被看见的属性,这些丝线构成了对于进入设施的人员的严重的安全威胁。在收容设施
之中留下的有机材料被观察到以大约4kg每天的速率消失,这是该生物是杂食食性的一个证据。

这些由SCP-183编织而出的丝线对于人类体验来说有着独到之处。测试证明了它们很可能是一种碳纤维管的形式,它们经常地出现,并且一出现后就紧紧绷在了墙壁、地板以及天花板之
间,没有表现出什么明显的模式,形成了一个复杂的、锐利的缠结状态。这似乎类似于蜘蛛的网,一个来缠住食物的陷阱。在房间之中的小动物没有受到惊扰直到它们因自己的行动而
死亡。如果它们没有被这些丝线造成致命伤,SCP-183将会吃掉脱落的身体部分但并不会攻击动物本身直到它们自己死亡。每一条丝线在数小时之后就消失了;我们提出的工作原理是这
些丝线失去了它们的粘合性后掉落在地,并且被SCP-183摄食之后进行了再利用。

有趣的是,SCP-183会以一种确定无疑的音乐模式来拨动它的丝线,创造出一种令人吃惊的复杂而悦耳的乐音,这种乐音可以以通常的五声音阶进行再次表示。这些有限的音符数目显示
出这些丝线是按照一定的长度和松紧度进行创造的,很有可能这显示出SCP-183有着高等智能。这有可能是为了吸引猎物;尝试着在相同的键位上重新弹奏这些曲调或是其它的声音并没
有导致SCP-183的其他活动,这证明了这些乐音并不是为了交流而弹奏的。

现在关于SCP-183的研究正主要集中在如何逆向操纵其机体的织丝机制和查明其隐形的机制。
