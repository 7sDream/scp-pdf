\chapter[SCP-167 无尽迷宫]{
    SCP-167 Infinite Labyrinth\\
    SCP-167 无尽迷宫
}

\label{chap:SCP-167}

\bb{项目编号:}SCP-167

\bb{项目等级:}Safe

\bb{特殊收容措施:}SCP-167现被保存在研究指挥部-06(Research Command-06)的██房内。房门在任何情况下均须挂锁,除非SCP-167正接受研究调查。任何想进行计划外的探索或者研究SCP-167的人必须获得负责该SCP的3级人员的许可方能拿到钥匙。

\bb{描述:}SCP-167是一个边长约为十米的立方体,由一种未知的、有光泽度的白色塑料聚合物组成。一扇大金属门附在立方体的一个面上。不清楚这扇门是该SCP本来就带有的,还是SCP基金会得到此物体之前被人装上去的。这个立方体的内部尺寸和外部大致相同,边长减小了几厘米,但是余下的三面墙\footnote{\bb{译注:}立方体六个面除去顶部、底部及有金属门的那一面后剩下的三个面}中的两面上开有门。这些门都是通向一模一样的房间,而每个房间更有两扇门通向更多相同的房间。在研究小组能够探明的范围内,该模式一直在重复。这些门道的位置似乎是随机的;还没有发现能够说明三面墙壁中哪两面有门的规律。

SCP-167有被探索过的迹象——一些房间,尤其是不太深的房间,在通往入口的门道旁画有红点或其它记号。最近,在文件\#167-08所述的事件之后,研究员也开始养成对他们到过的房间做这样的标记的习惯了。另外,已找到一些散落在SCP-167内的房间里的人造和天然的物品:像大约公元前500年的宗教人偶;几个公元1500年的宝箱;{[}数据删除];一些SCP,特别是{[}数据删除]。

\bb{附录167-01:}SCP-167似乎是不遵守欧几里得几何定理的,两个研究人员走在应该通向相同的房间的不同路径上,但进了房间后两个人谁也没见到谁,更无法听见对方的声音。尚不清楚SCP-167是如何改变空间来达到这一效果的,值得进一步的研究考察。

\bb{附录167-02:}有人提议测试\hyperref[chap:SCP-184]{SCP-184}对SCP-167的影响,此请求正在考虑当中。

\bb{附录167-03:}█████博士建议用SCP-167作为一个紧凑的储存空间用来储存一些良性的SCP。这项建议需要将SCP-167重分级为Safe,因此即将进行一次重新评估。

\bb{文件\#167-08:}\ii{正如你们大多数人知道的,几天前摄像头拍到██████博士没带必需的麻线球就进入了SCP-167,到现在都没有回来。他最后怎么样了谁也不知道,但是搜索小组什么都没找到。希望这能提醒你们所有人,如果不用什么办法标记走过的路,你们很容易就会在里面迷路。如果我发现任何其他研究员违反安全规定,不带麻线球就进去了的话,不管他们想走进去多深,他们都会发现自己被调到了Keter级SCP研究设施。在那种地方他们应该有足够的动机去尽快学会遵守安全规定。}——Klein博士
