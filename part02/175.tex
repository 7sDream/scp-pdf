\chapter[SCP-175 宝藏图]{
    SCP-175 Treasure Map\\
    SCP-175 宝藏图
}

\label{chap:SCP-175}

\bb{项目编号:}SCP-175

\bb{项目等级:}Safe

\bb{特殊收容措施:}在未对其进行试验时,SCP-175保存于一个金属保险箱中。该保险箱同其他低保管等级SCP一同保存于███区仓库中。依据{[}数据删除]安全条款,该仓库由至少两名安保人员全天候看守。

\bb{描述:}当非“激活”时,SCP-175外观呈现为一张泛黄的羊皮纸,比常规纸张尺寸略大。边缘有裂痕,并具有遭受风化而易碎的特征。但实际上SCP-175具有与其外观不相称的柔韧性,且完全不会受到任何外界损伤。所有试图分割SCP-175纸体的举动皆告失败。

当附近存在埋藏起来的物体时,SCP-175将进入激活状态,并展现出真正特殊之处。SCP-175能够被激活的距离会因某些未知因素改变:从30米至最远数公里不等。多次试验得出的平均值为100-200米。

当SCP-175被(上述条件)“激活”时,其外观会变化并呈现为一张指向物品埋藏点的地图或其他类似的文字或表格。SCP-175在“激活”后不会改变其尺寸大小,但外观会因寻物者或物品埋藏者的思维模式而改变,改变内容包括书写媒介,书写工具以及书写风格的变化。

将所指示的埋藏物被挖出,或将项目从埋藏地附近移开后,SCP-175会还原为原本的空白羊皮纸状态。

\bb{档案编号175-08:}埋藏者,埋藏物和SCP-175呈现外观的部分记录。若无特别注明,所有试验都于{[}数据删除]处进行。

███████博士埋藏了一个木匣。SCP-175激活后呈现为一张坐标图,以铅笔标注出周围区域的环境,且于结尾处标注有比例尺。埋藏地点坐标位置被清楚标明。███████博士评论该实验中SCP-175上出现的字迹与他的笔迹几乎完全一致。

████ ██████——█████博士的5岁女儿——埋藏了一个木匣,内装埋藏者拥有的数件玩具。SCP-175激活后呈现为一张蜡笔画,在白纸上画出埋藏点附近的景物。作画风格符合该年龄段的普通孩童画风。

███████ ████——特工█████10岁的儿子——埋藏了装有其漫画书的盒子。SCP-175激活后呈现为一张黄底表格,列有该地区的显著标志物(例如树和石头)。表格内有详细挖掘指示:包括出发点,前进步数,于何处转弯,以及最后目的地即挖掘点。指示中有部分拼写错误,符合该年龄段孩童知识水平。

████ ████████——富有威望的风景画画家——埋藏了一个空木匣。SCP-175激活,外形呈现为一张埋藏地点的鸟瞰视角油画,画中于埋藏点标注有X记号。

█████████ ███████——业内谜题大师,擅长设计纵横拼字谜——埋藏了一个木匣。SCP-175激活后呈现为一张油墨印制的方格纸,上面包括由数个纵横拼字谜和其他类型谜题一同组成的复合字谜。只有通过解出所有其他的谜题才能解出该复合字谜,复合字谜的答案为找出埋藏物品的线索。

特工█████忆起其在8岁时曾经将一个装着许多小玩意的盒子埋于自家后院。随后他与研究小组一同赶往{[}数据删除]处。SCP-175激活后呈现为一张铅笔画,画出该处周围的环境。已证实其笔迹不符合该特工现在的书写习惯,而与他父母保存下来的他于8岁时的书写笔迹相同。值得一提的是,地图上还有数棵现实已不存在的树。这些树木于近年间被砍伐,但SCP-175依旧将其标示出来。

D级人员——前建筑设计师,连环杀手——被指示埋藏一个木匣。{[}数据删除]。关于该次实验的详细完整的记录可在获得高级批准后查阅附录175-13。

在SCP-175连同其他若干件等级为Safe的其他SCP一同被运往{[}数据删除]的过程中,一名卡车上的安保人员报告,装有SCP-175的箱内传出一个高分贝的哀嚎声。依据规定,该卡车立刻停止运输并回撤。在得悉箱内物品资料后,最靠近该处的机动特遣队Omega-7(潘多拉之盒)前往处理该事件。根据任务汇报,特遣队发现声音来源为箱中一片金属薄片。其发出的哀嚎,音调和音量会因靠近某特定位置而提高。随即在该地挖掘出SCP-███,而该金属片也即时复原为常态下的SCP-175。

\bb{附录175-13:}于███████。一名D级人员——前建筑设计师,连环杀手——被指示埋藏一个木匣;该实验本意为验证SCP-175在埋藏者为建筑师时会展现何种外观。埋藏结束后,██████博士开始查看SCP-175,随即大声尖叫,跪倒在地,双手紧紧抱头。SCP-175背面朝上丢弃于地面,此乃不幸中之万幸。据实验过程中与██████博士站在同侧的特工████汇报,在SCP-175被丢掉前,他瞥见其上布满大量闪烁不定的色块,并使其感到极度恶心。随后他马上将木匣挖出,已还原SCP-175。███████博士于事故后陷入永久性昏迷。经严格心理审查后显示该D级人员有潜在的精神分裂和反社会倾向。自该实验后,已设立了严格的安全制度以保证研究员的精神健康。
