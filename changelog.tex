% 版本历史

% 开发版本使用 .dev 后缀
% PR 提交者需要在最新 dev 版的参与人员(第三个大括号)内
% 加上自己的代号,使用 | 分割,例如 QIS|AB|CC

% 首次提交 PR 的参与人员需要在 editors.tex 里添加自己的代号和昵称

\begin{versionhistory}
\addcontentsline{toc}{part}{版本历史}
\vhEntry
{0.1}{2014.3.30}{QIS}{
	初次发布,完成SCP-001的部分提案和SCP-002;
}
\vhEntry
{1.0}{2017.5.7}{QIS}{
	完全重做,翻译源从论坛更改至Wiki;
	增加了版本历史,更新新加入的SCP-001提案;
	整理 \TeX 源文件并在Github上\href{https://github.com/7sDream/scp-pdf}{开源}
}
\vhEntry
{1.1}{2017.5.8}{QIS}{
	修改了SCP-001原型里的符号错误;
	增加了SCP-003\textasciitilde SCP-010
}
\vhEntry
{1.2}{2017.5.8}{QIS}{
	修改了前文里的几个脚注;
	增加了SCP-011\textasciitilde SCP-020
}
\vhEntry
{1.3}{2017.5.12}{QIS}{
	校对了SCP-001中的文章;
	修改封面;
	去除生成PDF文章标题上的间隔;
	增加Kindle专版6寸PDF;
	增加了SCP-021\textasciitilde SCP-030
}
\vhEntry
{1.4}{2017.5.13}{QIS}{
    增加了SCP-031\textasciitilde SCP-040
}
\vhEntry
{1.5}{2017.5.14}{QIS}{
    校对前文;
    修正了Kindle版本上文字上方注释的问题;
    修正了Kindle版本SCP-001 黎明将至中对话框超过版面的问题;
    增加了SCP-041\textasciitilde SCP-050
}
\vhEntry
{1.6}{2017.5.14}{QIS}{
    增加了SCP-051\textasciitilde SCP-060
}
\vhEntry
{1.7}{2017.5.14}{QIS}{
    增加了SCP-061\textasciitilde SCP-070
}
\vhEntry
{1.9}{2017.5.16}{QIS}{
    增加了合集部分,目前只收录了传承合集;
    修改传承合集的版面样式;
    增加了SCP-071\textasciitilde SCP-090
}
\vhEntry
{1.10}{2017.5.16}{QIS}{
    修复1.9版本中故事部分缺失的问题;
    将补充文档移动到相关SCP本体附近,便于阅读;
    增加了SCP-091\textasciitilde SCP-100
}
\vhEntry
{1.11}{2017.6.17}{QIS}{
    修复1.10版本中SCP-075无标题的问题;
    增加了SCP-101\textasciitilde SCP-110
}
\vhEntry
{1.12}{2017.7.11}{QIS}{
    按照要求,本文档内容和源码均使用CC-BY-SA授权协议;
    将全文繁体的篇目都转换为简体;
    现在每个附加文档都新起一页;
    现在Caption默认居中;
    增加了SCP-111\textasciitilde SCP-120
}
\vhEntry
{1.13}{2017.7.12}{QIS}{
    增加了SCP-121\textasciitilde SCP-130
}
\vhEntry
{1.14}{2017.7.14}{QIS}{
    增加了SCP-131\textasciitilde SCP-140
}
\vhEntry
{1.15}{2017.8.12}{QIS}{
    将英文等宽字体从SF Mono切换至Fira Code,因为SF Mono的许可协议暂不明确;
    增加了SCP-141\textasciitilde SCP-1150
}
\vhEntry
{1.16}{2017.8.19}{QIS}{
    增加了SCP-151\textasciitilde SCP-160
    增加了奥林匹亚项目相关资料
}
\vhEntry
{1.17}{2018.2.24}{QIS}{
    增加了不同字体的版本,目前有Noto和Sarasa两个版本;
    更新了SCP-113的翻译;
    SCP-001 When day breaks 标题从“黎明将至”重翻译为“破晓之时”;
    SCP-001的目录样式修改,和wiki上保持一致;
    增加新SCP-001提案《上帝的盲点(God's Blind Spot)》;
    增加新SCP-001提案《常态(Normalcy)》;
    使用xeCJKfntef宏包代替ulem宏包,获得更好的删除线下划线效果;
    PC版本的overtextnote注释字体缩小;
    删除了SCP-001提案《一致意见(The Consensus)》标题中的修复脚注;
    增加了SCP-161\textasciitilde SCP-170
}
\vhEntry
{1.18.dev}{\today}{QIS}{
    增加新SCP-001提案《死人(Dead Men)》;
    修正文字绿色过于鲜艳的问题;
    修正了几个SCP-001标题中缺少的CODENAME的问题;
    增加了适用于一加3的编译版本,理论上16:9的手机均适用;
    更新SCP-150;
    修复SCP-178附加文档里粗体和下划线同时使用时只有第一个字显示为粗体的问题;
    内部调整,现在不使用 for loop 来 input 各 SCP 文件,而是直接 input 每一个,对编辑软件更友好;
    sarasa 配置中文也支持同时含有粗体和斜体的情况;
    增加新SCP-001提案《世界,收容失败(The World at Large)》;
    增加新SCP-001提案《赎罪(Atonement)》;
    增加新SCP-001提案《好孩子(A Good Boy)》;
    跟随网页版调整,将《孩子们》《破碎之神》《赎罪》《终结的方式》四篇合并为《响尾蛇》,但《终结的方式》目前还未翻译完成,暂未收录;
    增加新SCP-001 提案《O5-13》;
    增加新SCP-001 提案《鱼钩》
}
\end{versionhistory}

如果发现任何错误或者对本文档有建议,请\href{https://github.com/7sDream/scp-pdf/issues}{提交Issue}。
