% 文档里的特殊字符使用特定字体显示
% 如果需要增加字体,需要在下列「定义特殊字符备用字体」一节中增加

% 使用方法:\<font>{<char>},文档中的 <char> 字符使用 <font> 字体显示。

% 特殊字符单独支持
\usepackage{newunicodechar}

% 定义特殊字符备用字体
\newfontfamily{\emojifont}{Noto Emoji}
\DeclareTextFontCommand{\useemoji}{\emojifont}
\newcommand{\emoji}[1]{\newunicodechar{#1}{\useemoji{#1}}}

\newfontfamily{\cjkmainfont}{\targetcjkmainfont}
\DeclareTextFontCommand{\usecjkmain}{\cjkmainfont}
\newcommand{\cjkmain}[1]{\newunicodechar{#1}{\usecjkmain{#1}}}

\newfontfamily{\djvsfont}{DejaVu Sans}
\DeclareTextFontCommand{\usedjvs}{\djvsfont}
\newcommand{\djvs}[1]{\newunicodechar{#1}{\usedjvs{#1}}}

\newfontfamily{\freeseriffont}[Extension=.otf]{FreeSerif}
\DeclareTextFontCommand{\usefreeserif}{\freeseriffont}
\newcommand{\freeserif}[1]{\newunicodechar{#1}{\usefreeserif{#1}}}

\xeCJKDeclareSubCJKBlock{SIP}
{
	"20000 -> "2A6DF , % CJK Unified Ideographs Extension B
	"2A700 -> "2B73F , % CJK Unified Ideographs Extension C
	"2B740 -> "2B81F   % CJK Unified Ideographs Extension D
}

\cjkmain{█}

% SCP-001.when.day.breaks
\emoji{🔥}

% SCP-001.the.factory
\cjkmain{⋯}

% SCP-001.the.children
\freeserif{❚}

% SCP-001.a.record
\djvs{ת}
\cjkmain{▷}
\cjkmain{▽}
\cjkmain{⦿}

% SCP-001.normalcy 更*多*报告-仅限O5阅读
\setCJKmainfont[SIP]{HanaMinB}

% SCP-111
\cjkmain{℃}

% SCP-116
\cjkmain{≧}

% SCP-150
\cjkmain{Ⅲ}

% SCP-185
\cjkmain{─}
