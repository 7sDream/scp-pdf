% 快捷命令

% 段落间空行
\newcommand{\vs}{\vspace{12pt}}
\newcommand{\vsl}{\vspace{24pt}}

% 文本颜色
\newcommand{\red}{\textcolor{red}}
\newcommand{\green}{\textcolor{green}}
\newcommand{\tred}{\textcolor{textred}}
\newcommand{\wred}{\textcolor{warningred}}
\newcommand{\gsix}{\textcolor{graysix}}

% 文本粗体斜体各种线
\newcommand{\ii}{\textit}
\newcommand{\bb}{\textbf}
\newcommand{\bi}[1]{\textit{\textbf{#1}}}
\newcommand{\uu}{\CJKunderline}
\newcommand{\dd}{\CJKsout}
\newcommand{\bu}[1]{\CJKunderline{\textbf{#1}}}
\newcommand{\bd}[1]{\textbf{\CJKsout{#1}}}

% 文本等宽
\newcommand{\mono}[1]{{\ttfamily #1}}

% 居中
\newcommand{\cl}[1]{\begin{center}#1\end{center}}

% 字体大小
\newcommand{\g}[1]{{\large #1}}
\newcommand{\Gg}[1]{{\Large #1}}
\newcommand{\GG}[1]{{\LARGE #1}}
\newcommand{\Hg}[1]{{\Huge #1}}

% URL
\renewcommand{\url}[1]{\href{#1}{\nolinkurl{#1}}}

% hrule
\newcommand{\hr}{
\noindent\hfil
\textcolor{boxrulecolor}{\rule{0.8\textwidth}{0.4pt}}
\hfil
}

% 文本正上方标注
\makeatletter
\def\overtextnote#1{\expandafter\overtextnote@aux#1\@nil}
\ifthenelse{\equal{\targetdevice}{pc}}{
	\def\overtextnote@aux#1(#2)\@nil{\renewcommand{\arraystretch}{0.5}\begin{tabular}[b]{@{}c@{}}#2\\#1\end{tabular}\renewcommand{\arraystretch}{1.5}}
}{
    \def\overtextnote@aux#1(#2)\@nil{#1({#2})}
}
\makeatother

% 传承条目

\newcommand{\heritage}{\hyperref[chap:COMP-heritage]{\includegraphics[height=\fontcharht\font`\B]{images/heritage.png} \tred{传承条目}}}
